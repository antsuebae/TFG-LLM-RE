\cdpchapter{Resumen}

Esta documentaci\'on corresponde a un proyecto de final de carrera consistente en la creaci\'on de una clase \LaTeX{} llamada
\texttt{pclass}, dicha clase tiene como finalidad el formateo de memorias de proyectos pertenecientes a la Escuela T\'ecnica Superior de Ingenier\'ia Informa\'tica. A pesar de ello con algunas modificaciones no resultar\'a complicado adaptar esta plantilla para aplicarla al resto de titulaciones.

A lo largo de esta memoria se detallar\'a el procedimiento seguido para la creaci\'on partiendo de cero de \texttt{pclass}.
Adem\'as de este proceso de creaci\'on, tambi\'en se contemplar\'an las distintas funcionalidades propias de la clase,
las cuales facilitar\'an de manera m\'as que notoria la redacci\'on de una memoria.

Finalmente si lo que te interesa es pasar directamente a redactar el contenido de tu memoria, puedes omitir los distintos
pasos seguidos para crear \texttt{pclass} y dirigirte al Cap\'itulo~\ref{manual}. En este cap\'itulo puedes encontrar un
manual completo de uso de la clase que da origen a este proyecto.




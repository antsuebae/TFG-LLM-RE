\chapter{Introducci\'on}\label{intro}

\section{¿Qu\'e es \TeX{}?}

	El creador de Tex es Donald E. Knuth, su trabajo fue un encargo de la American Mathematical Society a principios de los a\~nos 70. Esta sociedad buscaba un lenguaje para formatear sus art\'iculos llenos de teoremas y f\'ormulas matem\'aticas de gran complejidad. El resulatdo obtenido fue un lenguaje extremadamente potente, pero tambi\'en dif\'icil de aprender y usar. 

	Basta decir que de hecho el sistema \LaTeX{} es el estandar de creaci\'on de textos cient\'ificos desde hace muchos a\~nos, sin embargo aprender \LaTeX{} no es cosa de un d\'ia, no es algo f\'acil pero tampoco imposible, de forma que con algo de paciencia se pueden conseguir resultados casi inmediatos. Para faclitar el trabajo con \TeX{} surgieron numerosas macros que agrupaban distintas instrucciones de \TeX{}.\\
				
	\subsection{¿Qu\'e es \LaTeX{}?}
		\LaTeX{} es un paquete de macros, especialmente dise\~nado para la creaci\'on de textos t\'ecnicos y cient\'ificos, que permite componer e imprimir documentos de forma sencilla, con la mayor calidad tipogr\'afica, utilizando para ello patrones previamente definidos. Est\'a basado en un lenguaje de composici\'on de bajo nivel llamado \TeX{} y facilita el uso de este potente lenguaje. 
		
			A diferencia de otros sistemas para procesar textos, no se obtiene el resultado final a medida que se va escribiendo sino que primero se crea un c\'odigo fuente y seguidamente se procesa para llegar al documento final, en este sentido se asemeja mucho a los lenguajes de marcas como el HTML.
		
		\LaTeX{} fue escrito por Leslie Lamport en los a\~nos 80 y actualmente multitud de libros, revistas cient\'ificas 
		est\'an escritas integramente en \LaTeX{}, incluso en numerosos foros cient\'ificos se ha convertido en el 
		est\'andar exigido para la publicaci\'on de resultados. Una de las razones de la gran difusi\'on de \LaTeX{} es su 
		precio. \LaTeX{} es freeware, es decir, puede conseguirse a trav\'es de internet y utilizarse de forma gratuita y 
		legal. Sin embargo la ventaja fundamental entre \LaTeX{} y otros procesadores m\'as conocidos (Word Perfect, 
		Microsoft Word) es la calidad de los documentos que genera, fundamantalmente cuando aparecen involucrados textos 
		que incluyen numerosas f\'ormulas matem\'aticas, ecuaciones, tablas, etc. Adem\'as est\'a disponible para la 
		pr\'actica totalidad de sistemas operativos actuales, incluyendo Windows, Linux, Unix ,etc. Otra de sus ventajas es 
		la existencia de una gran cantidad de paquetes est\'andares pensados para dotar a los textos de toda la funcionalidad 
		que se precise. As\'i existen paquetes para incluir gr\'aficos, textos de lenguajes de programación, f\'ormulas 
		f\'isicas y qu\'imicas, diagramas matem\'aticos, etc. Por todo ello \LaTeX{} ha conocido una gran difusi\'on en el 
		\'ambito cient\'ifico, siendo hoy d\'ia el procesador m\'as usado por matem\'aticos, f\'isicos y gran n\'umero 
		de ingenieros.\\
		
			\figura{0.6}{img/knuth}{Donald E.Knuth}{img:knuth}{}
			 
		 
\subsection{¿Qu\'e es \LaTeXe{} }
		Revisi\'on completa desde la versi\'on \LaTeX{} 2.09, que fue durante muchos a\~nos la versi\'on estandard de \LaTeX{} hasta la apoarici\'on de \LaTeXe{}, uno de sus prop\'ositos centrales fue la integarci\'on dentro de un ambiente \'unico de \LaTeX{}. La idea fundamental de \LaTeXe{} es que toda futura adici\'on o extensi\'on de \LaTeX{} se haga por medio de paquetes individuales, que el usuario puede invocar por medio de la instrucci\'on \verb+\usepackage{...}+. De este modo se intenta poner fin a la proliferaci\'on de dialectos incompatibles.  
		
	

\section{\LaTeX{} vs WYSIWYG}
	¿Qui\'en no ha enviado un documento escrito con un procesador de textos cl\'asico a una impresora diferente de la de su ordenador y ha obtenido un  resultado desastroso, incluyendo cambio de fuentes, modificaci\'on de la paginaci\'on, etc.?. Todo esto es historia con \LaTeX{}. Digamos que en \LaTeX{}, el usuario se concentra en la estructura l\'ogica del documento m\'as que en su apariencia, ya que \'esta se define aparte. Ello permite modificar de forma r\'apida y eficaz la apariencia, sin modificar en absoluto el contenido.
	    
		\LaTeX{} desempe\~na el papel de dise\~nador tomando parte en el formato del documento (longitud del rengl\'on, tipo de letra, espacios, \ldots) para darle luego instrucciones al cajista, \TeX{}.	El tratmiento del texto es totalmente diferente a procesadores tales como Microsoft Word o Word Perfect en los cuales el autor ve en pantalla lo que exactamente aparecer\'a luego en la impresora. Esto tiene sus ventajas e inconvenientes como comentaremos m\'as adelante.
		
		Se le dar\'a mayor importancia a la legibilidad y comprensi\'on del texto que al aspecto m\'as o menos agradable que este pueda presentar. Con un sistema WYSIWYG \footnote{Siglas que significan, What you see is what you get, lo que ve es lo que obtendr\'a.} podemos obtener textos est\'eticamente bonitos pero con una estructura muy peque\~na o inconsistente. Sin embargo con \LaTeX{} esto no est\'a permitido ya que el autor est\'a forzado a seguir un orden e indicar una estructura.

\subsection{Ventajas e inconvenientes de \LaTeX{}}%sacado de manual latex de universidad de cadiz
Ventajas:

\begin{itemize}
	\item Facilita la composici\'on de f\'ormulas con un cuidado especial. 
	\item Existe mayor cantidad de dise\~nos de textos profesionales a disposici\'on, con lo que realmente se pueden crear  documentos como si fueran de imprenta.
	\item No hace falta preocuparse por los detalles. S\'olo es necesario introducir instrucciones para indicar la estructura del documento.
	\item Las estructuras, tales como notas al pie de p\'agina, bibliograf\'ia, \'indices, tablas y muchas otras, pueden ser introducidas sin demasiado esfuerzo. 
	\item Existen paquetes adicionales, sin coste alguno, para muchas tareas tipogr\'aficas aunque no se facilitan directamente por el \LaTeX{} b\'asico.
	\item \LaTeX{} hace que los autores tiendan a escribir textos bien estructurados porque as\'i es como trabaja \LaTeX{}, o sea, indicando su estructura.
	\item \TeX{} es altamente portable y gratis. Por eso, el sistema funciona pr\'acticamente en cualquier plataforma.
\end{itemize}


Inconvenientes:

\begin{itemize} 
	\item Si bien se pueden ajustar algunos par\'ametros de un dise\~no de documento predefinido, la creaci\'on de un dise\~no entero es dif\'icil y lleva mucho tiempo. 
	\item El periodo de aprendizaje es mayor que los WYSIWYG.	
	\item No sirve para maquetaci\'on de publicaciones. Se necesita
invertir demasiado tiempo.
\end{itemize}